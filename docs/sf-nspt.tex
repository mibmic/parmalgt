\documentclass[a4paper]{scrartcl}

\usepackage[utf8]{inputenc}
\usepackage{amsmath}
\usepackage{gnuplot-lua-tikz}

\title{NSPT in the Schrödinger Functional}

\author{Dirk Hesse}

\date{\today}

\newcommand{\ord}[1]{\ensuremath{^{(#1)}}}

\begin{document}

\maketitle

We pursue the implementation of numerical stochastic perturbation
theory in the Schrödinger functional.

\section{Preliminary}
\label{sec:preliminary}

\subsection{Links With a Background Field}
\label{sec:links-with-backgr}

Assume we have a background field $V$ and fluctuations $q$, such that
we can write the links as
%
\begin{equation}
  \label{eq:1}
  U = \exp\{q\} V.
\end{equation}
%
We will use two different ways to express the perturbative expansion
of (\ref{eq:1}),
%
\begin{align}
  \label{eq:3}
  U =&\; V + g\, U\ord 1 + g^2\, U\ord 2 + \ldots\\
  \label{eq:5}
    =&\; \left (
      1 + g\, \tilde U\ord 1 + g^2\, \tilde U\ord 2 + \ldots
      \right) V \equiv \tilde U \,V.
\end{align}
%
The connection to the perturbative expansion of the algebra
valued $q$-field,
%
\begin{equation}
  \label{eq:2}
  q = g\, q\ord 1 + g^2 q \ord 2 + \ldots
\end{equation}
%
can be established writing (\ref{eq:1}) explicitly,
%
\begin{equation}
  \label{eq:4}
  \exp \{ q \} = 1 + g \, q \ord 1 + g^2\, \left\{
    q \ord 2 + \frac 1 2 \left(q\ord 1\right)^2 \right\}
   + \ldots,
\end{equation}
%
and identifying the terms in the series with the ones in
(\ref{eq:5}) order by order. To go from the group to the algebra, one
has to calculate
%
\begin{equation}
  \label{eq:6}
  q = \log\{\tilde U\} = g\, \tilde U \ord 1 + g^2 \, \left\{
    \tilde U \ord 2 - \frac 1 2 \left( \tilde U \ord 1 \right)^2
    \right \} + \ldots,
\end{equation}
%
and compare the terms with the naive series
%
\begin{equation}
  \label{eq:7}
  q = g\,q\ord 1 + g^2 \,q\ord 2 + \ldots
\end{equation}
%
again, order by order in the coupling.

\subsubsection{Arithmetic}
\label{sec:arithmetic}

The multiplication rule for the representation (\ref{eq:3}) is rather
simple, we get (setting $U\ord 0 \equiv V$ and $\tilde U \ord 0 \equiv
1$)
%
\begin{equation}
  \label{eq:8}
  \left ( U' \times U \right)\ord i = \sum_{j = 0}^{i} {U'}\ord j U
  \ord{i-j} = \sum_{j = 0}^{i} {\tilde U '}\hspace{0em}\ord j V' 
  {\tilde U} \ord {i- j}V.
\end{equation}
%
Finally, we have
%
\begin{equation}
  \label{eq:9}
  \widetilde {\left( U' \times U \right)}\ord i = \left ( U' \times U
  \right)\ord i \left(V'\times V \right)^{-1}.
\end{equation}
%
For the addition, we have the following rules,
%
\begin{equation}
  \label{eq:10}
  \left ( U' + U \right)\ord i = {U'}\ord i + U \ord i = {\tilde U'}
  \hspace{0em}\ord i V' + {\tilde U}\ord i V,
\end{equation}
%
and again, we must perform
%
\begin{equation}
  \label{eq:11}
  \widetilde {\left( U' + U\right)}\ord i = \left ( U' + U
  \right)\ord i \left(V' + V \right)^{-1}.
\end{equation}

\subsubsection{Implementation}
\label{sec:implementation}

Inspecting the addition and multiplication rules given in the last
section, it seems beneficial to store the link variables in the
expansion (\ref{eq:3}). Only for applying the exponential function,
the form (\ref{eq:5}) is needed in priciple.

\subsubsection{Exponential and Logarithm}
\label{sec:expon-logar}

We opt for the following convention. The \texttt{BGptSU3} type lives
always in the Group $SU(3)$. Another type, \texttt{PtMatrix} (defined
for now in \texttt{PtTypes.hpp}) is responsible for storing elements
of the algebra $su(3)$. The $g$-expansion represented by
\texttt{PtMatrix} is always assumed to start with $g$. The methods
$\log$ and \texttt{reH} are understood as mappings $SU(3) \to su(3)$,
and assume that the background field is unity.

If the natural logarithm $\log$ is applied to a \texttt{BGptSU3}
instance, the method will assume that the background field associated
with its arguments is unity. It will calculate $q = \log \{\tilde U\}$
using the Mercator series
%
\begin{equation}
  \label{eq:13}
  \log \{\tilde U\} = \sum_{i=1}^{\mathrm{PT\_ORD}} 
  (-1)^{n+1} \frac {(\tilde U - 1)^n} n,
\end{equation}
%
aborting the series at the highest perturbative order considered
(i.e. \texttt{PT\_ORD} in the code). If the user calls
%
\begin{equation}
  \label{eq:19}
  \log(V + U),
\end{equation}
%
where $V + U$ is represented by a \texttt{BGptSU3} instance, the
program will check if $V = 1$ holds and throw an exception if it is
not the case. The same check is in place for
%
\begin{equation}
  \label{eq:20}
  \mathrm{reH}(V + U) = \frac 1 2 \left[ (V + U) - (V + U)^\dagger
    \right]_{\mathrm{traceless}}.
\end{equation}
%
When the exponential function is applied, the code assumes that we
have stored the algebra valued fields $q$ in the \texttt{PtMatrix}
instance given as the argument of $\exp$. The exponential
is then evaluated in the usual way,
%
\begin{equation}
  \label{eq:14}
  \tilde U = \exp \{ q\} = \sum_{i = 0}^{\mathrm{PT\_ORD}} \frac {q^n}{n!},
\end{equation}
%
and finally the $U\ord i = {\tilde U}\ord i V$ are returned. Since the
perturbative expansion of $q$ is assumed to start at $g$, we can set
$V = 1$.

\section{Boundary Conditions}
\label{sec:boundary-conditions}

The Schrödinger functional boundary conditions are implemented with
relative ease. We have

\begin{equation}
  \label{eq:15}
  U_k(x) = W_k(\vec x)|_{x_0 = 0}, \quad  U_k(x) = W_k'(\vec x)|_{x_0 = T}.
\end{equation}
%
The boundary conditions in the spatial directrions are periodic, and
thus the existing code does not need to be changed. In the current
code, the temporal size of the allocated storage is called
\texttt{act\_pars.zs[3]}, we will refer to this variable as $s_3$ for
brevity. A rather simple way to implement the Dirchlet boundary
conditions is to identify $s_3 \equiv (T + a)/a$. When the lattice is
sweeped in the gauge update, the lattice site is identified by an
integer value \texttt{curr}. We will refer to this label with $n$. It
is calculated as
%
\begin{equation}
  \label{eq:16}
  n = x_0 + s_3 ( x_1 + L [ x_2 + L\, x_3] ).
\end{equation}
%
Hence, we can easily extract the time component using $t = (n \mod
s_3)$. Then we just abort the gauge update if we hit the boundary,
%
\begin{equation}
  \label{eq:17}
  (t = 0 \land \mu \neq 0) \lor t = T\quad \Rightarrow
  \mathtt{abort}.
\end{equation}

\section{Gauge Fixing, Zero Mode Subtraction}
\label{sec:gauge-fixing}

Here, we present the results of some experimentation with gauge fixing
and zero mode subtraction. A very simplistic (and ultimately wrong)
approach is to just use the Faddeev-Popov method for all active
links. This is, as is pointed out in \cite{Luscher:1992an}, is however
not an admissible procedure. In our calculations, we found that the
norm of the gauge field increases abnormally and thus indicates that
the simulation is invalid, c.f. figure \ref{fig:norm} (the approach
described above is labeled "Plain Faddeev-Popov ").

\begin{figure}
  \centering
  \begin{tikzpicture}[gnuplot]
%% generated with GNUPLOT 4.4p2 (Lua 5.1.4; terminal rev. 97, script rev. 96a)
%% Wed 08 Feb 2012 12:05:10 PM CET
\tikzstyle{every node}+=[font=\small]
\gpmonochromelines
\gpcolor{gp lt color border}
\gpsetlinetype{gp lt border}
\gpsetlinewidth{1.00}
\draw[gp path] (2.240,0.985)--(2.420,0.985);
\draw[gp path] (11.947,0.985)--(11.767,0.985);
\node[gp node right] at (2.056,0.985) { 0};
\draw[gp path] (2.240,1.910)--(2.420,1.910);
\draw[gp path] (11.947,1.910)--(11.767,1.910);
\node[gp node right] at (2.056,1.910) { 200000};
\draw[gp path] (2.240,2.834)--(2.420,2.834);
\draw[gp path] (11.947,2.834)--(11.767,2.834);
\node[gp node right] at (2.056,2.834) { 400000};
\draw[gp path] (2.240,3.759)--(2.420,3.759);
\draw[gp path] (11.947,3.759)--(11.767,3.759);
\node[gp node right] at (2.056,3.759) { 600000};
\draw[gp path] (2.240,4.683)--(2.420,4.683);
\draw[gp path] (11.947,4.683)--(11.767,4.683);
\node[gp node right] at (2.056,4.683) { 800000};
\draw[gp path] (2.240,5.608)--(2.420,5.608);
\draw[gp path] (11.947,5.608)--(11.767,5.608);
\node[gp node right] at (2.056,5.608) { 1e+06};
\draw[gp path] (2.240,6.532)--(2.420,6.532);
\draw[gp path] (11.947,6.532)--(11.767,6.532);
\node[gp node right] at (2.056,6.532) { 1.2e+06};
\draw[gp path] (2.240,7.457)--(2.420,7.457);
\draw[gp path] (11.947,7.457)--(11.767,7.457);
\node[gp node right] at (2.056,7.457) { 1.4e+06};
\draw[gp path] (2.240,8.381)--(2.420,8.381);
\draw[gp path] (11.947,8.381)--(11.767,8.381);
\node[gp node right] at (2.056,8.381) { 1.6e+06};
\draw[gp path] (2.240,0.985)--(2.240,1.165);
\draw[gp path] (2.240,8.381)--(2.240,8.201);
\node[gp node center] at (2.240,0.677) { 0};
\draw[gp path] (3.211,0.985)--(3.211,1.165);
\draw[gp path] (3.211,8.381)--(3.211,8.201);
\node[gp node center] at (3.211,0.677) { 1000};
\draw[gp path] (4.181,0.985)--(4.181,1.165);
\draw[gp path] (4.181,8.381)--(4.181,8.201);
\node[gp node center] at (4.181,0.677) { 2000};
\draw[gp path] (5.152,0.985)--(5.152,1.165);
\draw[gp path] (5.152,8.381)--(5.152,8.201);
\node[gp node center] at (5.152,0.677) { 3000};
\draw[gp path] (6.123,0.985)--(6.123,1.165);
\draw[gp path] (6.123,8.381)--(6.123,8.201);
\node[gp node center] at (6.123,0.677) { 4000};
\draw[gp path] (7.094,0.985)--(7.094,1.165);
\draw[gp path] (7.094,8.381)--(7.094,8.201);
\node[gp node center] at (7.094,0.677) { 5000};
\draw[gp path] (8.064,0.985)--(8.064,1.165);
\draw[gp path] (8.064,8.381)--(8.064,8.201);
\node[gp node center] at (8.064,0.677) { 6000};
\draw[gp path] (9.035,0.985)--(9.035,1.165);
\draw[gp path] (9.035,8.381)--(9.035,8.201);
\node[gp node center] at (9.035,0.677) { 7000};
\draw[gp path] (10.006,0.985)--(10.006,1.165);
\draw[gp path] (10.006,8.381)--(10.006,8.201);
\node[gp node center] at (10.006,0.677) { 8000};
\draw[gp path] (10.976,0.985)--(10.976,1.165);
\draw[gp path] (10.976,8.381)--(10.976,8.201);
\node[gp node center] at (10.976,0.677) { 9000};
\draw[gp path] (11.947,0.985)--(11.947,1.165);
\draw[gp path] (11.947,8.381)--(11.947,8.201);
\node[gp node center] at (11.947,0.677) { 10000};
\draw[gp path] (2.240,8.381)--(2.240,0.985)--(11.947,0.985)--(11.947,8.381)--cycle;
\node[gp node center,rotate=-270] at (0.246,4.683) {$\|U^{(12)(x)}\|$};
\node[gp node center] at (7.093,0.215) {$	au$};
\node[gp node right] at (10.479,8.047) {Naive GF, with zero mode sub.};
\gpcolor{gp lt color 0}
\gpsetpointsize{4.00}
\gppoint{gp mark 5}{(2.240,0.985)}
\gppoint{gp mark 5}{(2.434,1.180)}
\gppoint{gp mark 5}{(2.628,1.292)}
\gppoint{gp mark 5}{(2.822,1.407)}
\gppoint{gp mark 5}{(3.017,1.388)}
\gppoint{gp mark 5}{(3.211,1.373)}
\gppoint{gp mark 5}{(3.405,1.383)}
\gppoint{gp mark 5}{(3.599,1.488)}
\gppoint{gp mark 5}{(3.793,1.538)}
\gppoint{gp mark 5}{(3.987,1.405)}
\gppoint{gp mark 5}{(4.181,1.360)}
\gppoint{gp mark 5}{(4.376,1.390)}
\gppoint{gp mark 5}{(4.570,1.480)}
\gppoint{gp mark 5}{(4.764,1.508)}
\gppoint{gp mark 5}{(4.958,1.488)}
\gppoint{gp mark 5}{(5.152,1.444)}
\gppoint{gp mark 5}{(5.346,1.419)}
\gppoint{gp mark 5}{(5.540,1.435)}
\gppoint{gp mark 5}{(5.735,1.444)}
\gppoint{gp mark 5}{(5.929,1.462)}
\gppoint{gp mark 5}{(6.123,1.486)}
\gppoint{gp mark 5}{(6.317,1.447)}
\gppoint{gp mark 5}{(6.511,1.471)}
\gppoint{gp mark 5}{(6.705,1.423)}
\gppoint{gp mark 5}{(6.899,1.400)}
\gppoint{gp mark 5}{(7.094,1.433)}
\gppoint{gp mark 5}{(7.288,1.426)}
\gppoint{gp mark 5}{(7.482,1.406)}
\gppoint{gp mark 5}{(7.676,1.388)}
\gppoint{gp mark 5}{(7.870,1.416)}
\gppoint{gp mark 5}{(8.064,1.436)}
\gppoint{gp mark 5}{(8.258,1.479)}
\gppoint{gp mark 5}{(8.452,1.473)}
\gppoint{gp mark 5}{(8.647,1.418)}
\gppoint{gp mark 5}{(8.841,1.460)}
\gppoint{gp mark 5}{(9.035,1.388)}
\gppoint{gp mark 5}{(9.229,1.398)}
\gppoint{gp mark 5}{(9.423,1.411)}
\gppoint{gp mark 5}{(9.617,1.459)}
\gppoint{gp mark 5}{(9.811,1.399)}
\gppoint{gp mark 5}{(10.006,1.382)}
\gppoint{gp mark 5}{(10.200,1.381)}
\gppoint{gp mark 5}{(10.394,1.522)}
\gppoint{gp mark 5}{(10.588,1.550)}
\gppoint{gp mark 5}{(10.782,1.396)}
\gppoint{gp mark 5}{(10.976,1.382)}
\gppoint{gp mark 5}{(11.170,1.504)}
\gppoint{gp mark 5}{(11.365,1.445)}
\gppoint{gp mark 5}{(11.559,1.438)}
\gppoint{gp mark 5}{(11.753,1.452)}
\gppoint{gp mark 5}{(11.121,8.047)}
\gpcolor{gp lt color border}
\node[gp node right] at (10.479,7.739) {Naive GF, without zero mode sub.};
\gpcolor{gp lt color 1}
\gppoint{gp mark 7}{(2.240,0.985)}
\gppoint{gp mark 7}{(2.289,0.989)}
\gppoint{gp mark 7}{(2.337,1.065)}
\gppoint{gp mark 7}{(2.386,1.131)}
\gppoint{gp mark 7}{(2.434,1.186)}
\gppoint{gp mark 7}{(2.483,1.227)}
\gppoint{gp mark 7}{(2.531,1.284)}
\gppoint{gp mark 7}{(2.580,1.296)}
\gppoint{gp mark 7}{(2.628,1.307)}
\gppoint{gp mark 7}{(2.677,1.324)}
\gppoint{gp mark 7}{(2.725,1.378)}
\gppoint{gp mark 7}{(2.774,1.426)}
\gppoint{gp mark 7}{(2.822,1.462)}
\gppoint{gp mark 7}{(2.871,1.426)}
\gppoint{gp mark 7}{(2.919,1.418)}
\gppoint{gp mark 7}{(2.968,1.437)}
\gppoint{gp mark 7}{(3.017,1.450)}
\gppoint{gp mark 7}{(3.065,1.482)}
\gppoint{gp mark 7}{(3.114,1.478)}
\gppoint{gp mark 7}{(3.162,1.494)}
\gppoint{gp mark 7}{(3.211,1.501)}
\gppoint{gp mark 7}{(3.259,1.477)}
\gppoint{gp mark 7}{(3.308,1.484)}
\gppoint{gp mark 7}{(3.356,1.486)}
\gppoint{gp mark 7}{(3.405,1.507)}
\gppoint{gp mark 7}{(3.453,1.546)}
\gppoint{gp mark 7}{(3.502,1.577)}
\gppoint{gp mark 7}{(3.550,1.584)}
\gppoint{gp mark 7}{(3.599,1.591)}
\gppoint{gp mark 7}{(3.648,1.550)}
\gppoint{gp mark 7}{(3.696,1.567)}
\gppoint{gp mark 7}{(3.745,1.591)}
\gppoint{gp mark 7}{(3.793,1.602)}
\gppoint{gp mark 7}{(3.842,1.608)}
\gppoint{gp mark 7}{(3.890,1.627)}
\gppoint{gp mark 7}{(3.939,1.579)}
\gppoint{gp mark 7}{(3.987,1.517)}
\gppoint{gp mark 7}{(4.036,1.518)}
\gppoint{gp mark 7}{(4.084,1.530)}
\gppoint{gp mark 7}{(4.133,1.501)}
\gppoint{gp mark 7}{(4.181,1.484)}
\gppoint{gp mark 7}{(4.230,1.483)}
\gppoint{gp mark 7}{(4.278,1.485)}
\gppoint{gp mark 7}{(4.327,1.522)}
\gppoint{gp mark 7}{(4.376,1.526)}
\gppoint{gp mark 7}{(4.424,1.523)}
\gppoint{gp mark 7}{(4.473,1.584)}
\gppoint{gp mark 7}{(4.521,1.595)}
\gppoint{gp mark 7}{(4.570,1.605)}
\gppoint{gp mark 7}{(4.618,1.596)}
\gppoint{gp mark 7}{(4.667,1.635)}
\gppoint{gp mark 7}{(4.715,1.665)}
\gppoint{gp mark 7}{(4.764,1.684)}
\gppoint{gp mark 7}{(4.812,1.684)}
\gppoint{gp mark 7}{(4.861,1.694)}
\gppoint{gp mark 7}{(4.909,1.721)}
\gppoint{gp mark 7}{(4.958,1.679)}
\gppoint{gp mark 7}{(5.006,1.668)}
\gppoint{gp mark 7}{(5.055,1.661)}
\gppoint{gp mark 7}{(5.104,1.673)}
\gppoint{gp mark 7}{(5.152,1.669)}
\gppoint{gp mark 7}{(5.201,1.609)}
\gppoint{gp mark 7}{(5.249,1.613)}
\gppoint{gp mark 7}{(5.298,1.590)}
\gppoint{gp mark 7}{(5.346,1.619)}
\gppoint{gp mark 7}{(5.395,1.646)}
\gppoint{gp mark 7}{(5.443,1.664)}
\gppoint{gp mark 7}{(5.492,1.675)}
\gppoint{gp mark 7}{(5.540,1.660)}
\gppoint{gp mark 7}{(5.589,1.630)}
\gppoint{gp mark 7}{(5.637,1.685)}
\gppoint{gp mark 7}{(5.686,1.730)}
\gppoint{gp mark 7}{(5.735,1.747)}
\gppoint{gp mark 7}{(5.783,1.819)}
\gppoint{gp mark 7}{(5.832,1.875)}
\gppoint{gp mark 7}{(5.880,1.896)}
\gppoint{gp mark 7}{(5.929,1.822)}
\gppoint{gp mark 7}{(5.977,1.815)}
\gppoint{gp mark 7}{(6.026,1.818)}
\gppoint{gp mark 7}{(6.074,1.869)}
\gppoint{gp mark 7}{(6.123,1.886)}
\gppoint{gp mark 7}{(6.171,1.904)}
\gppoint{gp mark 7}{(6.220,1.849)}
\gppoint{gp mark 7}{(6.268,1.801)}
\gppoint{gp mark 7}{(6.317,1.816)}
\gppoint{gp mark 7}{(6.365,1.893)}
\gppoint{gp mark 7}{(6.414,1.919)}
\gppoint{gp mark 7}{(6.463,1.935)}
\gppoint{gp mark 7}{(6.511,1.946)}
\gppoint{gp mark 7}{(6.560,1.931)}
\gppoint{gp mark 7}{(6.608,1.954)}
\gppoint{gp mark 7}{(6.657,1.924)}
\gppoint{gp mark 7}{(6.705,1.898)}
\gppoint{gp mark 7}{(6.754,1.910)}
\gppoint{gp mark 7}{(6.802,1.888)}
\gppoint{gp mark 7}{(6.851,1.889)}
\gppoint{gp mark 7}{(6.899,1.873)}
\gppoint{gp mark 7}{(6.948,1.884)}
\gppoint{gp mark 7}{(6.996,1.932)}
\gppoint{gp mark 7}{(7.045,1.923)}
\gppoint{gp mark 7}{(7.094,1.928)}
\gppoint{gp mark 7}{(7.142,1.921)}
\gppoint{gp mark 7}{(7.191,1.930)}
\gppoint{gp mark 7}{(7.239,1.933)}
\gppoint{gp mark 7}{(7.288,1.935)}
\gppoint{gp mark 7}{(7.336,1.968)}
\gppoint{gp mark 7}{(7.385,1.940)}
\gppoint{gp mark 7}{(7.433,1.952)}
\gppoint{gp mark 7}{(7.482,2.016)}
\gppoint{gp mark 7}{(7.530,1.952)}
\gppoint{gp mark 7}{(7.579,1.925)}
\gppoint{gp mark 7}{(7.627,1.897)}
\gppoint{gp mark 7}{(7.676,1.930)}
\gppoint{gp mark 7}{(7.724,1.948)}
\gppoint{gp mark 7}{(7.773,1.973)}
\gppoint{gp mark 7}{(7.822,1.981)}
\gppoint{gp mark 7}{(7.870,1.979)}
\gppoint{gp mark 7}{(7.919,1.995)}
\gppoint{gp mark 7}{(7.967,2.011)}
\gppoint{gp mark 7}{(8.016,2.024)}
\gppoint{gp mark 7}{(8.064,2.111)}
\gppoint{gp mark 7}{(8.113,2.116)}
\gppoint{gp mark 7}{(8.161,2.067)}
\gppoint{gp mark 7}{(8.210,2.067)}
\gppoint{gp mark 7}{(8.258,2.145)}
\gppoint{gp mark 7}{(8.307,2.179)}
\gppoint{gp mark 7}{(8.355,2.163)}
\gppoint{gp mark 7}{(8.404,2.216)}
\gppoint{gp mark 7}{(8.452,2.245)}
\gppoint{gp mark 7}{(8.501,2.263)}
\gppoint{gp mark 7}{(8.550,2.293)}
\gppoint{gp mark 7}{(8.598,2.287)}
\gppoint{gp mark 7}{(8.647,2.342)}
\gppoint{gp mark 7}{(8.695,2.422)}
\gppoint{gp mark 7}{(8.744,2.507)}
\gppoint{gp mark 7}{(8.792,2.543)}
\gppoint{gp mark 7}{(8.841,2.544)}
\gppoint{gp mark 7}{(8.889,2.534)}
\gppoint{gp mark 7}{(8.938,2.489)}
\gppoint{gp mark 7}{(8.986,2.569)}
\gppoint{gp mark 7}{(9.035,2.606)}
\gppoint{gp mark 7}{(9.083,2.691)}
\gppoint{gp mark 7}{(9.132,2.800)}
\gppoint{gp mark 7}{(9.181,2.900)}
\gppoint{gp mark 7}{(9.229,2.967)}
\gppoint{gp mark 7}{(9.278,3.101)}
\gppoint{gp mark 7}{(9.326,3.267)}
\gppoint{gp mark 7}{(9.375,3.407)}
\gppoint{gp mark 7}{(9.423,3.524)}
\gppoint{gp mark 7}{(9.472,3.500)}
\gppoint{gp mark 7}{(9.520,3.490)}
\gppoint{gp mark 7}{(9.569,3.529)}
\gppoint{gp mark 7}{(9.617,3.469)}
\gppoint{gp mark 7}{(9.666,3.521)}
\gppoint{gp mark 7}{(9.714,3.666)}
\gppoint{gp mark 7}{(9.763,3.624)}
\gppoint{gp mark 7}{(9.811,3.939)}
\gppoint{gp mark 7}{(9.860,4.336)}
\gppoint{gp mark 7}{(9.909,4.609)}
\gppoint{gp mark 7}{(9.957,4.659)}
\gppoint{gp mark 7}{(10.006,4.839)}
\gppoint{gp mark 7}{(10.054,5.053)}
\gppoint{gp mark 7}{(10.103,5.293)}
\gppoint{gp mark 7}{(10.151,5.375)}
\gppoint{gp mark 7}{(10.200,5.496)}
\gppoint{gp mark 7}{(10.248,5.728)}
\gppoint{gp mark 7}{(10.297,5.949)}
\gppoint{gp mark 7}{(10.345,6.284)}
\gppoint{gp mark 7}{(10.394,6.524)}
\gppoint{gp mark 7}{(10.442,6.755)}
\gppoint{gp mark 7}{(11.121,7.739)}
\gpcolor{gp lt color border}
\node[gp node right] at (10.479,7.431) {Plain Faddeev-Popov};
\gpcolor{gp lt color 2}
\gppoint{gp mark 9}{(2.240,0.985)}
\gppoint{gp mark 9}{(2.242,0.985)}
\gppoint{gp mark 9}{(2.244,0.985)}
\gppoint{gp mark 9}{(2.246,0.985)}
\gppoint{gp mark 9}{(2.248,0.985)}
\gppoint{gp mark 9}{(2.250,0.985)}
\gppoint{gp mark 9}{(2.252,0.985)}
\gppoint{gp mark 9}{(2.254,0.985)}
\gppoint{gp mark 9}{(2.256,0.985)}
\gppoint{gp mark 9}{(2.257,0.985)}
\gppoint{gp mark 9}{(2.259,0.985)}
\gppoint{gp mark 9}{(2.261,0.985)}
\gppoint{gp mark 9}{(2.263,0.985)}
\gppoint{gp mark 9}{(2.265,0.985)}
\gppoint{gp mark 9}{(2.267,0.985)}
\gppoint{gp mark 9}{(2.269,0.985)}
\gppoint{gp mark 9}{(2.271,0.985)}
\gppoint{gp mark 9}{(2.273,0.986)}
\gppoint{gp mark 9}{(2.275,0.986)}
\gppoint{gp mark 9}{(2.277,0.986)}
\gppoint{gp mark 9}{(2.279,0.986)}
\gppoint{gp mark 9}{(2.281,0.987)}
\gppoint{gp mark 9}{(2.283,0.987)}
\gppoint{gp mark 9}{(2.285,0.988)}
\gppoint{gp mark 9}{(2.287,0.988)}
\gppoint{gp mark 9}{(2.289,0.989)}
\gppoint{gp mark 9}{(2.290,0.990)}
\gppoint{gp mark 9}{(2.292,0.991)}
\gppoint{gp mark 9}{(2.294,0.992)}
\gppoint{gp mark 9}{(2.296,0.993)}
\gppoint{gp mark 9}{(2.298,0.995)}
\gppoint{gp mark 9}{(2.300,0.997)}
\gppoint{gp mark 9}{(2.302,0.999)}
\gppoint{gp mark 9}{(2.304,1.002)}
\gppoint{gp mark 9}{(2.306,1.004)}
\gppoint{gp mark 9}{(2.308,1.007)}
\gppoint{gp mark 9}{(2.310,1.010)}
\gppoint{gp mark 9}{(2.312,1.013)}
\gppoint{gp mark 9}{(2.314,1.017)}
\gppoint{gp mark 9}{(2.316,1.020)}
\gppoint{gp mark 9}{(2.318,1.024)}
\gppoint{gp mark 9}{(2.320,1.028)}
\gppoint{gp mark 9}{(2.322,1.032)}
\gppoint{gp mark 9}{(2.323,1.036)}
\gppoint{gp mark 9}{(2.325,1.040)}
\gppoint{gp mark 9}{(2.327,1.045)}
\gppoint{gp mark 9}{(2.329,1.049)}
\gppoint{gp mark 9}{(2.331,1.053)}
\gppoint{gp mark 9}{(2.333,1.057)}
\gppoint{gp mark 9}{(2.335,1.060)}
\gppoint{gp mark 9}{(2.337,1.064)}
\gppoint{gp mark 9}{(2.339,1.067)}
\gppoint{gp mark 9}{(2.341,1.070)}
\gppoint{gp mark 9}{(2.343,1.072)}
\gppoint{gp mark 9}{(2.345,1.075)}
\gppoint{gp mark 9}{(2.347,1.077)}
\gppoint{gp mark 9}{(2.349,1.080)}
\gppoint{gp mark 9}{(2.351,1.082)}
\gppoint{gp mark 9}{(2.353,1.085)}
\gppoint{gp mark 9}{(2.355,1.087)}
\gppoint{gp mark 9}{(2.356,1.089)}
\gppoint{gp mark 9}{(2.358,1.092)}
\gppoint{gp mark 9}{(2.360,1.094)}
\gppoint{gp mark 9}{(2.362,1.097)}
\gppoint{gp mark 9}{(2.364,1.099)}
\gppoint{gp mark 9}{(2.366,1.101)}
\gppoint{gp mark 9}{(2.368,1.104)}
\gppoint{gp mark 9}{(2.370,1.106)}
\gppoint{gp mark 9}{(2.372,1.109)}
\gppoint{gp mark 9}{(2.374,1.112)}
\gppoint{gp mark 9}{(2.376,1.115)}
\gppoint{gp mark 9}{(2.378,1.118)}
\gppoint{gp mark 9}{(2.380,1.122)}
\gppoint{gp mark 9}{(2.382,1.125)}
\gppoint{gp mark 9}{(2.384,1.128)}
\gppoint{gp mark 9}{(2.386,1.131)}
\gppoint{gp mark 9}{(2.388,1.135)}
\gppoint{gp mark 9}{(2.389,1.138)}
\gppoint{gp mark 9}{(2.391,1.142)}
\gppoint{gp mark 9}{(2.393,1.145)}
\gppoint{gp mark 9}{(2.395,1.149)}
\gppoint{gp mark 9}{(2.397,1.153)}
\gppoint{gp mark 9}{(2.399,1.157)}
\gppoint{gp mark 9}{(2.401,1.160)}
\gppoint{gp mark 9}{(2.403,1.163)}
\gppoint{gp mark 9}{(2.405,1.166)}
\gppoint{gp mark 9}{(2.407,1.169)}
\gppoint{gp mark 9}{(2.409,1.172)}
\gppoint{gp mark 9}{(2.411,1.174)}
\gppoint{gp mark 9}{(2.413,1.177)}
\gppoint{gp mark 9}{(2.415,1.179)}
\gppoint{gp mark 9}{(2.417,1.182)}
\gppoint{gp mark 9}{(2.419,1.185)}
\gppoint{gp mark 9}{(2.421,1.188)}
\gppoint{gp mark 9}{(2.422,1.190)}
\gppoint{gp mark 9}{(2.424,1.193)}
\gppoint{gp mark 9}{(2.426,1.195)}
\gppoint{gp mark 9}{(2.428,1.200)}
\gppoint{gp mark 9}{(2.430,1.204)}
\gppoint{gp mark 9}{(2.432,1.208)}
\gppoint{gp mark 9}{(2.434,1.213)}
\gppoint{gp mark 9}{(2.436,1.218)}
\gppoint{gp mark 9}{(2.438,1.226)}
\gppoint{gp mark 9}{(2.440,1.238)}
\gppoint{gp mark 9}{(2.442,1.243)}
\gppoint{gp mark 9}{(2.444,1.256)}
\gppoint{gp mark 9}{(2.446,1.266)}
\gppoint{gp mark 9}{(2.448,1.277)}
\gppoint{gp mark 9}{(2.450,1.293)}
\gppoint{gp mark 9}{(2.452,1.313)}
\gppoint{gp mark 9}{(2.454,1.332)}
\gppoint{gp mark 9}{(2.455,1.351)}
\gppoint{gp mark 9}{(2.457,1.378)}
\gppoint{gp mark 9}{(2.459,1.404)}
\gppoint{gp mark 9}{(2.461,1.420)}
\gppoint{gp mark 9}{(2.463,1.456)}
\gppoint{gp mark 9}{(2.465,1.503)}
\gppoint{gp mark 9}{(2.467,1.542)}
\gppoint{gp mark 9}{(2.469,1.598)}
\gppoint{gp mark 9}{(2.471,1.644)}
\gppoint{gp mark 9}{(2.473,1.693)}
\gppoint{gp mark 9}{(2.475,1.746)}
\gppoint{gp mark 9}{(2.477,1.765)}
\gppoint{gp mark 9}{(2.479,1.826)}
\gppoint{gp mark 9}{(2.481,1.854)}
\gppoint{gp mark 9}{(2.483,1.903)}
\gppoint{gp mark 9}{(2.485,2.013)}
\gppoint{gp mark 9}{(2.487,2.045)}
\gppoint{gp mark 9}{(2.488,2.128)}
\gppoint{gp mark 9}{(2.490,2.182)}
\gppoint{gp mark 9}{(2.492,2.228)}
\gppoint{gp mark 9}{(2.494,2.302)}
\gppoint{gp mark 9}{(2.496,2.386)}
\gppoint{gp mark 9}{(2.498,2.395)}
\gppoint{gp mark 9}{(2.500,2.444)}
\gppoint{gp mark 9}{(2.502,2.578)}
\gppoint{gp mark 9}{(2.504,2.657)}
\gppoint{gp mark 9}{(2.506,2.687)}
\gppoint{gp mark 9}{(2.508,2.887)}
\gppoint{gp mark 9}{(2.510,3.005)}
\gppoint{gp mark 9}{(2.512,3.176)}
\gppoint{gp mark 9}{(2.514,3.357)}
\gppoint{gp mark 9}{(2.516,3.452)}
\gppoint{gp mark 9}{(2.518,3.711)}
\gppoint{gp mark 9}{(2.520,3.863)}
\gppoint{gp mark 9}{(2.522,4.079)}
\gppoint{gp mark 9}{(2.523,4.342)}
\gppoint{gp mark 9}{(2.525,4.359)}
\gppoint{gp mark 9}{(2.527,4.557)}
\gppoint{gp mark 9}{(2.529,4.934)}
\gppoint{gp mark 9}{(2.531,5.264)}
\gppoint{gp mark 9}{(2.533,5.595)}
\gppoint{gp mark 9}{(2.535,5.424)}
\gppoint{gp mark 9}{(2.537,5.767)}
\gppoint{gp mark 9}{(2.539,6.163)}
\gppoint{gp mark 9}{(2.541,6.331)}
\gppoint{gp mark 9}{(2.543,6.524)}
\gppoint{gp mark 9}{(2.545,6.837)}
\gppoint{gp mark 9}{(2.547,6.894)}
\gppoint{gp mark 9}{(2.549,7.088)}
\gppoint{gp mark 9}{(2.551,7.392)}
\gppoint{gp mark 9}{(2.553,7.548)}
\gppoint{gp mark 9}{(2.555,7.801)}
\gppoint{gp mark 9}{(2.556,8.045)}
\gppoint{gp mark 9}{(2.558,8.314)}
\gppoint{gp mark 9}{(11.121,7.431)}
\gpcolor{gp lt color border}
\draw[gp path] (2.240,8.381)--(2.240,0.985)--(11.947,0.985)--(11.947,8.381)--cycle;
%% coordinates of the plot area
\gpdefrectangularnode{gp plot 1}{\pgfpoint{2.240cm}{0.985cm}}{\pgfpoint{11.947cm}{8.381cm}}
\end{tikzpicture}
%% gnuplot variables
  
  \caption{History of the norm of $U\ord {12}$ on
    a $T/a = 5$, $L/a = 6$ lattice.}
  \label{fig:norm}
\end{figure}

The problem at hand is that only the diagonal modes of the temporal
components of $q$ should be fixed at the $t=0$ boundary, and thus the
usual procedure is not valid. A naive approach to solve this is to
omit the gauge fixing of said field components altogether, hoping that
there will be some restoring force that prevents an abnormal raise of
the norm of the fields that are not fixed. This could come e.g. from
the boundary. This method seems to work, inspecting figure
\ref{fig:norm} again, we only find that a subtraction of zero modes is
required. The procedure explained above is labeled "Naive GF" in the
figure.

To investigate the issue of the temporal components of the gauge
fields at $t=0$ more closely, we define the norm

\begin{equation}
  \label{eq:18}
  n_0\ord i = \frac 1 {L^3} \sum_{\vec x}\left\| U_0 \ord i(0, \vec
    x) \right\|.
\end{equation}

Figure \ref{fig:n0nor} shows a sample history of $n_0\ord {12}$. We
see that the value stabilizes and we may thus infer that the naive
gauge fixing procedure mentioned above does not expose any unexpected
behavior.

\begin{figure}
  \centering
  \begin{tikzpicture}[gnuplot]
%% generated with GNUPLOT 4.4p2 (Lua 5.1.4; terminal rev. 97, script rev. 96a)
%% Wed 08 Feb 2012 12:23:54 PM CET
\tikzstyle{every node}+=[font=\small]
\gpmonochromelines
\gpcolor{gp lt color border}
\gpsetlinetype{gp lt border}
\gpsetlinewidth{1.00}
\draw[gp path] (1.504,0.985)--(1.684,0.985);
\draw[gp path] (11.793,0.985)--(11.613,0.985);
\node[gp node right] at (1.320,0.985) { 0};
\draw[gp path] (1.504,1.725)--(1.684,1.725);
\draw[gp path] (11.793,1.725)--(11.613,1.725);
\node[gp node right] at (1.320,1.725) { 50};
\draw[gp path] (1.504,2.464)--(1.684,2.464);
\draw[gp path] (11.793,2.464)--(11.613,2.464);
\node[gp node right] at (1.320,2.464) { 100};
\draw[gp path] (1.504,3.204)--(1.684,3.204);
\draw[gp path] (11.793,3.204)--(11.613,3.204);
\node[gp node right] at (1.320,3.204) { 150};
\draw[gp path] (1.504,3.943)--(1.684,3.943);
\draw[gp path] (11.793,3.943)--(11.613,3.943);
\node[gp node right] at (1.320,3.943) { 200};
\draw[gp path] (1.504,4.683)--(1.684,4.683);
\draw[gp path] (11.793,4.683)--(11.613,4.683);
\node[gp node right] at (1.320,4.683) { 250};
\draw[gp path] (1.504,5.423)--(1.684,5.423);
\draw[gp path] (11.793,5.423)--(11.613,5.423);
\node[gp node right] at (1.320,5.423) { 300};
\draw[gp path] (1.504,6.162)--(1.684,6.162);
\draw[gp path] (11.793,6.162)--(11.613,6.162);
\node[gp node right] at (1.320,6.162) { 350};
\draw[gp path] (1.504,6.902)--(1.684,6.902);
\draw[gp path] (11.793,6.902)--(11.613,6.902);
\node[gp node right] at (1.320,6.902) { 400};
\draw[gp path] (1.504,7.641)--(1.684,7.641);
\draw[gp path] (11.793,7.641)--(11.613,7.641);
\node[gp node right] at (1.320,7.641) { 450};
\draw[gp path] (1.504,8.381)--(1.684,8.381);
\draw[gp path] (11.793,8.381)--(11.613,8.381);
\node[gp node right] at (1.320,8.381) { 500};
\draw[gp path] (1.504,0.985)--(1.504,1.165);
\draw[gp path] (1.504,8.381)--(1.504,8.201);
\node[gp node center] at (1.504,0.677) { 0};
\draw[gp path] (2.533,0.985)--(2.533,1.165);
\draw[gp path] (2.533,8.381)--(2.533,8.201);
\node[gp node center] at (2.533,0.677) { 1000};
\draw[gp path] (3.562,0.985)--(3.562,1.165);
\draw[gp path] (3.562,8.381)--(3.562,8.201);
\node[gp node center] at (3.562,0.677) { 2000};
\draw[gp path] (4.591,0.985)--(4.591,1.165);
\draw[gp path] (4.591,8.381)--(4.591,8.201);
\node[gp node center] at (4.591,0.677) { 3000};
\draw[gp path] (5.620,0.985)--(5.620,1.165);
\draw[gp path] (5.620,8.381)--(5.620,8.201);
\node[gp node center] at (5.620,0.677) { 4000};
\draw[gp path] (6.649,0.985)--(6.649,1.165);
\draw[gp path] (6.649,8.381)--(6.649,8.201);
\node[gp node center] at (6.649,0.677) { 5000};
\draw[gp path] (7.677,0.985)--(7.677,1.165);
\draw[gp path] (7.677,8.381)--(7.677,8.201);
\node[gp node center] at (7.677,0.677) { 6000};
\draw[gp path] (8.706,0.985)--(8.706,1.165);
\draw[gp path] (8.706,8.381)--(8.706,8.201);
\node[gp node center] at (8.706,0.677) { 7000};
\draw[gp path] (9.735,0.985)--(9.735,1.165);
\draw[gp path] (9.735,8.381)--(9.735,8.201);
\node[gp node center] at (9.735,0.677) { 8000};
\draw[gp path] (10.764,0.985)--(10.764,1.165);
\draw[gp path] (10.764,8.381)--(10.764,8.201);
\node[gp node center] at (10.764,0.677) { 9000};
\draw[gp path] (11.793,0.985)--(11.793,1.165);
\draw[gp path] (11.793,8.381)--(11.793,8.201);
\node[gp node center] at (11.793,0.677) { 10000};
\draw[gp path] (1.504,8.381)--(1.504,0.985)--(11.793,0.985)--(11.793,8.381)--cycle;
\node[gp node center,rotate=-270] at (0.246,4.683) {$n_0^{(12)}$};
\node[gp node center] at (6.648,0.215) {$\tau$};
\gpcolor{gp lt color 0}
\gpsetpointsize{4.00}
\gppoint{gp mark 5}{(1.504,0.985)}
\gppoint{gp mark 5}{(1.555,1.078)}
\gppoint{gp mark 5}{(1.607,1.513)}
\gppoint{gp mark 5}{(1.658,1.945)}
\gppoint{gp mark 5}{(1.710,2.432)}
\gppoint{gp mark 5}{(1.761,2.824)}
\gppoint{gp mark 5}{(1.813,3.265)}
\gppoint{gp mark 5}{(1.864,3.619)}
\gppoint{gp mark 5}{(1.916,3.825)}
\gppoint{gp mark 5}{(1.967,4.117)}
\gppoint{gp mark 5}{(2.018,4.447)}
\gppoint{gp mark 5}{(2.070,4.834)}
\gppoint{gp mark 5}{(2.121,5.020)}
\gppoint{gp mark 5}{(2.173,5.204)}
\gppoint{gp mark 5}{(2.224,5.312)}
\gppoint{gp mark 5}{(2.276,5.404)}
\gppoint{gp mark 5}{(2.327,5.588)}
\gppoint{gp mark 5}{(2.379,5.830)}
\gppoint{gp mark 5}{(2.430,5.835)}
\gppoint{gp mark 5}{(2.481,5.904)}
\gppoint{gp mark 5}{(2.533,5.854)}
\gppoint{gp mark 5}{(2.584,5.904)}
\gppoint{gp mark 5}{(2.636,6.054)}
\gppoint{gp mark 5}{(2.687,6.352)}
\gppoint{gp mark 5}{(2.739,6.490)}
\gppoint{gp mark 5}{(2.790,6.590)}
\gppoint{gp mark 5}{(2.842,6.817)}
\gppoint{gp mark 5}{(2.893,6.890)}
\gppoint{gp mark 5}{(2.944,6.884)}
\gppoint{gp mark 5}{(2.996,6.853)}
\gppoint{gp mark 5}{(3.047,7.093)}
\gppoint{gp mark 5}{(3.099,7.127)}
\gppoint{gp mark 5}{(3.150,7.429)}
\gppoint{gp mark 5}{(3.202,7.361)}
\gppoint{gp mark 5}{(3.253,7.420)}
\gppoint{gp mark 5}{(3.305,7.290)}
\gppoint{gp mark 5}{(3.356,7.016)}
\gppoint{gp mark 5}{(3.407,7.152)}
\gppoint{gp mark 5}{(3.459,7.133)}
\gppoint{gp mark 5}{(3.510,7.136)}
\gppoint{gp mark 5}{(3.562,7.133)}
\gppoint{gp mark 5}{(3.613,7.300)}
\gppoint{gp mark 5}{(3.665,7.393)}
\gppoint{gp mark 5}{(3.716,7.401)}
\gppoint{gp mark 5}{(3.768,7.253)}
\gppoint{gp mark 5}{(3.819,7.263)}
\gppoint{gp mark 5}{(3.870,7.205)}
\gppoint{gp mark 5}{(3.922,7.289)}
\gppoint{gp mark 5}{(3.973,7.401)}
\gppoint{gp mark 5}{(4.025,7.481)}
\gppoint{gp mark 5}{(4.076,7.641)}
\gppoint{gp mark 5}{(4.128,7.648)}
\gppoint{gp mark 5}{(4.179,7.662)}
\gppoint{gp mark 5}{(4.231,7.637)}
\gppoint{gp mark 5}{(4.282,7.753)}
\gppoint{gp mark 5}{(4.333,7.532)}
\gppoint{gp mark 5}{(4.385,7.514)}
\gppoint{gp mark 5}{(4.436,7.407)}
\gppoint{gp mark 5}{(4.488,7.194)}
\gppoint{gp mark 5}{(4.539,7.226)}
\gppoint{gp mark 5}{(4.591,7.289)}
\gppoint{gp mark 5}{(4.642,7.158)}
\gppoint{gp mark 5}{(4.694,7.128)}
\gppoint{gp mark 5}{(4.745,7.117)}
\gppoint{gp mark 5}{(4.796,7.260)}
\gppoint{gp mark 5}{(4.848,7.080)}
\gppoint{gp mark 5}{(4.899,6.753)}
\gppoint{gp mark 5}{(4.951,6.593)}
\gppoint{gp mark 5}{(5.002,6.507)}
\gppoint{gp mark 5}{(5.054,6.679)}
\gppoint{gp mark 5}{(5.105,6.699)}
\gppoint{gp mark 5}{(5.157,6.651)}
\gppoint{gp mark 5}{(5.208,6.711)}
\gppoint{gp mark 5}{(5.259,6.746)}
\gppoint{gp mark 5}{(5.311,6.732)}
\gppoint{gp mark 5}{(5.362,6.879)}
\gppoint{gp mark 5}{(5.414,7.045)}
\gppoint{gp mark 5}{(5.465,7.167)}
\gppoint{gp mark 5}{(5.517,7.301)}
\gppoint{gp mark 5}{(5.568,7.143)}
\gppoint{gp mark 5}{(5.620,6.971)}
\gppoint{gp mark 5}{(5.671,6.859)}
\gppoint{gp mark 5}{(5.722,6.730)}
\gppoint{gp mark 5}{(5.774,6.944)}
\gppoint{gp mark 5}{(5.825,6.990)}
\gppoint{gp mark 5}{(5.877,7.092)}
\gppoint{gp mark 5}{(5.928,7.238)}
\gppoint{gp mark 5}{(5.980,7.148)}
\gppoint{gp mark 5}{(6.031,6.971)}
\gppoint{gp mark 5}{(6.083,7.222)}
\gppoint{gp mark 5}{(6.134,7.144)}
\gppoint{gp mark 5}{(6.185,6.939)}
\gppoint{gp mark 5}{(6.237,6.829)}
\gppoint{gp mark 5}{(6.288,6.845)}
\gppoint{gp mark 5}{(6.340,6.949)}
\gppoint{gp mark 5}{(6.391,7.117)}
\gppoint{gp mark 5}{(6.443,7.164)}
\gppoint{gp mark 5}{(6.494,7.068)}
\gppoint{gp mark 5}{(6.546,6.966)}
\gppoint{gp mark 5}{(6.597,6.879)}
\gppoint{gp mark 5}{(6.649,7.043)}
\gppoint{gp mark 5}{(6.700,7.019)}
\gppoint{gp mark 5}{(6.751,6.892)}
\gppoint{gp mark 5}{(6.803,6.945)}
\gppoint{gp mark 5}{(6.854,6.813)}
\gppoint{gp mark 5}{(6.906,6.913)}
\gppoint{gp mark 5}{(6.957,6.913)}
\gppoint{gp mark 5}{(7.009,6.792)}
\gppoint{gp mark 5}{(7.060,6.640)}
\gppoint{gp mark 5}{(7.112,6.839)}
\gppoint{gp mark 5}{(7.163,6.806)}
\gppoint{gp mark 5}{(7.214,6.697)}
\gppoint{gp mark 5}{(7.266,6.765)}
\gppoint{gp mark 5}{(7.317,6.716)}
\gppoint{gp mark 5}{(7.369,6.662)}
\gppoint{gp mark 5}{(7.420,6.786)}
\gppoint{gp mark 5}{(7.472,6.996)}
\gppoint{gp mark 5}{(7.523,6.950)}
\gppoint{gp mark 5}{(7.575,7.048)}
\gppoint{gp mark 5}{(7.626,6.832)}
\gppoint{gp mark 5}{(7.677,6.810)}
\gppoint{gp mark 5}{(7.729,6.809)}
\gppoint{gp mark 5}{(7.780,6.898)}
\gppoint{gp mark 5}{(7.832,6.810)}
\gppoint{gp mark 5}{(7.883,6.662)}
\gppoint{gp mark 5}{(7.935,6.525)}
\gppoint{gp mark 5}{(7.986,6.510)}
\gppoint{gp mark 5}{(8.038,6.575)}
\gppoint{gp mark 5}{(8.089,6.716)}
\gppoint{gp mark 5}{(8.140,6.636)}
\gppoint{gp mark 5}{(8.192,6.578)}
\gppoint{gp mark 5}{(8.243,6.509)}
\gppoint{gp mark 5}{(8.295,6.542)}
\gppoint{gp mark 5}{(8.346,6.500)}
\gppoint{gp mark 5}{(8.398,6.598)}
\gppoint{gp mark 5}{(8.449,6.627)}
\gppoint{gp mark 5}{(8.501,6.565)}
\gppoint{gp mark 5}{(8.552,6.573)}
\gppoint{gp mark 5}{(8.603,6.533)}
\gppoint{gp mark 5}{(8.655,6.608)}
\gppoint{gp mark 5}{(8.706,6.545)}
\gppoint{gp mark 5}{(8.758,6.346)}
\gppoint{gp mark 5}{(8.809,6.237)}
\gppoint{gp mark 5}{(8.861,6.199)}
\gppoint{gp mark 5}{(8.912,6.161)}
\gppoint{gp mark 5}{(8.964,6.258)}
\gppoint{gp mark 5}{(9.015,6.362)}
\gppoint{gp mark 5}{(9.066,6.468)}
\gppoint{gp mark 5}{(9.118,6.404)}
\gppoint{gp mark 5}{(9.169,6.481)}
\gppoint{gp mark 5}{(9.221,6.570)}
\gppoint{gp mark 5}{(9.272,6.665)}
\gppoint{gp mark 5}{(9.324,6.670)}
\gppoint{gp mark 5}{(9.375,6.627)}
\gppoint{gp mark 5}{(9.427,6.771)}
\gppoint{gp mark 5}{(9.478,6.979)}
\gppoint{gp mark 5}{(9.529,6.991)}
\gppoint{gp mark 5}{(9.581,6.941)}
\gppoint{gp mark 5}{(9.632,7.102)}
\gppoint{gp mark 5}{(9.684,7.119)}
\gppoint{gp mark 5}{(9.735,7.031)}
\gppoint{gp mark 5}{(9.787,7.186)}
\gppoint{gp mark 5}{(9.838,7.185)}
\gppoint{gp mark 5}{(9.890,7.101)}
\gppoint{gp mark 5}{(9.941,6.946)}
\gppoint{gp mark 5}{(9.992,7.014)}
\gppoint{gp mark 5}{(10.044,6.951)}
\gppoint{gp mark 5}{(10.095,7.125)}
\gppoint{gp mark 5}{(10.147,7.142)}
\gppoint{gp mark 5}{(10.198,7.115)}
\gppoint{gp mark 5}{(10.250,7.236)}
\gppoint{gp mark 5}{(10.301,7.083)}
\gppoint{gp mark 5}{(10.353,7.280)}
\gppoint{gp mark 5}{(10.404,7.342)}
\gppoint{gp mark 5}{(10.455,7.267)}
\gppoint{gp mark 5}{(10.507,7.317)}
\gppoint{gp mark 5}{(10.558,7.188)}
\gppoint{gp mark 5}{(10.610,7.209)}
\gppoint{gp mark 5}{(10.661,7.214)}
\gppoint{gp mark 5}{(10.713,7.156)}
\gppoint{gp mark 5}{(10.764,7.157)}
\gppoint{gp mark 5}{(10.816,6.918)}
\gppoint{gp mark 5}{(10.867,6.940)}
\gppoint{gp mark 5}{(10.918,6.855)}
\gppoint{gp mark 5}{(10.970,6.834)}
\gppoint{gp mark 5}{(11.021,6.901)}
\gppoint{gp mark 5}{(11.073,6.896)}
\gppoint{gp mark 5}{(11.124,6.969)}
\gppoint{gp mark 5}{(11.176,6.830)}
\gppoint{gp mark 5}{(11.227,7.122)}
\gppoint{gp mark 5}{(11.279,7.151)}
\gppoint{gp mark 5}{(11.330,7.090)}
\gppoint{gp mark 5}{(11.381,7.157)}
\gppoint{gp mark 5}{(11.433,7.122)}
\gppoint{gp mark 5}{(11.484,7.397)}
\gppoint{gp mark 5}{(11.536,7.495)}
\gppoint{gp mark 5}{(11.587,7.534)}
\gppoint{gp mark 5}{(11.639,7.576)}
\gppoint{gp mark 5}{(11.690,7.542)}
\gppoint{gp mark 5}{(11.742,7.304)}
\gpcolor{gp lt color border}
\draw[gp path] (1.504,8.381)--(1.504,0.985)--(11.793,0.985)--(11.793,8.381)--cycle;
%% coordinates of the plot area
\gpdefrectangularnode{gp plot 1}{\pgfpoint{1.504cm}{0.985cm}}{\pgfpoint{11.793cm}{8.381cm}}
\end{tikzpicture}
%% gnuplot variables

  \caption{History of $n_0\ord {12} $on a $T/a = 5$, 
    $L/a = 6$ lattice.}
  \label{fig:n0nor}
\end{figure}

\bibliographystyle{alpha}
\bibliography{bibliography.bib}

\end{document}
