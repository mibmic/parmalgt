\documentclass[a4paper]{article}

\usepackage[utf8]{inputenc}
\usepackage{amsmath}

\newcommand{\ct}{\ensuremath{c_t}}
\newcommand{\expv}[1]{\ensuremath{\langle{#1}\rangle}}
\newcommand{\expvz}[1]{\ensuremath{\langle{#1}\rangle_{\raisebox{-.25em}{\scriptsize 0}}}}

\title{Expectation values and the improvement coefficient \ct.}

\begin{document}

\maketitle

We work in the Schrödinger Functional scheme with action $S$ and
improvement coefficient \ct\ given as in \cite{Luscher:1992an}. We
have
%
\begin{equation}
  c_t = 1 + g_0^2 \dot c_t,\quad \dot c_t = c_t^{(1)} + g_0^2\,c_t^{(2)}.
\end{equation}
%
With that we can write the action as
%
\begin{equation}
  S(c_t) = S_0 + g_0^2 \,\dot c_t \,S_1 \equiv S_0 + \alpha\, S_1,
\end{equation}
%
and the partition function (from now on equations hold up to $O(\alpha^3)$)
%
\begin{align}
  Z &= \int \mathrm{d}U\; e^{-S(c_t)} \approx \int \mathrm{d}U\; e^{-S_0}
  \left( 1 - \alpha S_1 + \frac{\alpha^2}2 S_1^2 \right) \\
  &= Z_0 \left( 1 - \alpha
  \expvz{S_1} + \frac{\alpha^2}2 \expvz{S_1^2} \right).
\end{align}
% 
We introduced
%
\begin{equation}
  \expvz \cdot  \equiv \frac 1 {Z_0} \int \mathrm{d}U\; \cdot \,e^{-S_0}.
\end{equation}
%
We then get for a generic expectation value
%
\begin{align}
%  \nonumber
%  \expv{\cal O} = \frac 1 Z \int \mathrm{d}U\; {\cal O} \,e^{-S}\\
  \nonumber
  \expv{\cal O} = \frac 1 Z \int \mathrm{d}U\; {\cal O} \,e^{-S} \approx\;& \frac 1 {Z_0} \left( 1 + \alpha \, \expvz{S_1} +
    \alpha^2\left[\expvz{S_1}^2 - \frac {\expvz{S_1^2}} 2 \right]\right) \\
  \nonumber
  &\times \int \mathrm{d}U\; {\cal O} \,e^{-S_0} \left( 1  - \alpha
  \expvz{S_1} + \frac {\alpha^2}2 \expvz{S_1^2} \right)\\
\equiv \;&{\cal O}_0 + \alpha {\cal O}_1 + \alpha^2 {\cal
    O}_2 + O(\alpha^3)\,.
\end{align}
%
The coefficients are
%
\begin{align}
  {\cal O}_0 = \expvz{\cal O},\quad {\cal O}_1 = -\expvz{{\cal O} \,S_1}
  + \expvz{\cal O}\expvz{S_1}\,,\\
  {\cal O}_2 = \expvz{\cal O} \left( \expvz{S_1}^2 - \frac
    {\expvz{S_1^2}} 2
  \right)
  - \expvz{S_1} \expvz{{\cal O}\,S_1} + \frac{\expvz{{\cal
        O}\,S_1^2}} 2\,.
\end{align}
\begin{thebibliography}{99}
\bibitem{Luscher:1992an} 
  M.~Luscher, R.~Narayanan, P.~Weisz and U.~Wolff,
  %``The Schrodinger functional: A Renormalizable probe for nonAbelian gauge theories,''
  Nucl.\ Phys.\ B {\bf 384}, 168 (1992)
  [hep-lat/9207009].
  %%CITATION = HEP-LAT/9207009;%%
\end{thebibliography}

\end{document}
