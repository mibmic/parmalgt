\documentclass{report}

\usepackage{latexsym}
\usepackage[latin1]{inputenc}

\begin{document}
\section*{Quenched.cc e Unquenched.cc}
Piccola ``documentazione'' sulla struttura del codice. Per ora tratta solo
del programma principale, sia nella versione quenched
(\textbf{Quenched.cc}) che unqueched (\textbf{Quenched.cc}). Ho tenrato di renderlo il pi� possibile modulare, conforme alla struttura standard dei codici di QCD e facilmente leggibile\\
Schematicamente:\\
\\
\begin{tabular}{l}
\hline
\textbf{Corpo principale del programma}\\
\hline
\end{tabular}
\begin{description}
\item[initialize()]: inizializza i parametri per la simulazione,
  leggendoli da file o da riga di comando.
\item alloca il reticolo, il campo gluone ed eventualmente il campo fermione.
\item[QuenchedAllocate()/UnquenchedAllocate()]: inizializza il campo
  gluone, da freddo o da configurazione.
\item [for i = 1\ldots Sweep do]
\begin{description}
\item {}
\item[NsptEvolve()]: \emph{Sweep} passi di evoluzione NSPT
\item[AggiornaParametri()]: lettura del \emph{damocle} e correzione dei parametri
\end{description}
\item[end]
\item [NsptFinalize()]: chiude il logfile, salva il campo e dealloca tutto.
\end{description}
Tralasciando per un attimo l'inizializzazione dei parametri e l'allocazione dei campi (anche se questa seconda ha a che fare con la parallelizzazione), uno sguardo alla dinamica mostra questa strutturazione:\\
\\
\begin{tabular}{l}
\hline
\textbf{NsptEvolve}\\
\hline
\end{tabular}
\begin{description}
\item [for i = 1\ldots Beat do]
\begin{description}
\item {}
\item[gauge\_wilson()]: evoluzione di gauge, eventualmente improved, misura della placchetta e della norma
\item[fermion\_wilson()] (solo in caso unquenched): evoluzione della parte fermionica
\item[zero\_modes\_subtraction()]: sottrae il momento nullo al campo di gauge
\item [stochastic\_gauge\_fixing()]: fissaggio della gauge stocastica
\end{description}
\item[end]
\end{description}








\end{document}
